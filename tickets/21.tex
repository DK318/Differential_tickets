\textbf{Определение.} \textbf{Точкой покоя}, или \textbf{положением равновесия}, или \textbf{стационарным состоянием} системы $\dot{r} = f(r)$ называют точку $r_0$, такую что $f(r_0) = 0$.\\

\noindent \textbf{Определение.} Положение равновесия $r = 0$ автономной системы $\dot{r} = f(r)$ называется \textbf{усточивым (по Ляпунову)}, если для любой $\varepsilon$-окрестности нуля найдется такая $\delta$-окрестность нуля, что любое решение, выходящее из этой $\delta$-окрестности, во все будущие моменты времени отличается от нуля менее, чем на $\varepsilon$. То есть
\begin{equation*}
    \forall \varepsilon > 0 \, \exists \delta > 0:\, |r_0| < \delta \implies \forall t \ge 0 \, |r(t,0,r_0)| < \varepsilon
\end{equation*}
В противном случае положение равновесия называется \textbf{неустойчивым}.\\

\noindent \textbf{Определение.} Положение равновесия $r = 0$ автономной системы $\dot{r} = f(r)$ называется \textbf{асимптотически устойчивым}, если
\begin{itemize}
    \item $r = 0$ устойчиво
    \item все решения, начинающиеся в некоторой окрестности нуля, в будущем стремятся к нулю, то есть
    \begin{equation*}
        \exists \delta > 0:\, |r_0| < \delta \implies r(t,0,r_0) \to 0\, \text{при } t \to +\infty
    \end{equation*}
\end{itemize}

\noindent \textbf{Лемма} Пусть $\varphi$ --- решение системы $\dot{r} = f(t,r)$. Тогда $\varphi$ устойчиво (асимптотически устойчиво), если и только если устойчиво (асимптотически устойчиво) решение $s = 0$ системы
\begin{equation*}
    \dot{s} = f(t, s + \varphi) - f(t, \varphi)
\end{equation*}
\noindent \textbf{Доказательство.} Положим $s = r - \varphi$. Пусть $\dot{r} = f(t,r)$, тогда
\begin{equation*}
    \dot{s} = \dot{r} - \dot{\varphi} = f(t,r) - f(t,\varphi) = f(t, s + \varphi) - f(t,\varphi)
\end{equation*}
Получаем, что $s = 0$ --- решение системы $\dot{s} = f(t, s + \varphi) - f(t,\varphi)$. Остается сопоставить определение устойчивости (асимптотической устойчивости) для решения $\varphi$ исходной и решения $s = 0$ новой системы.\\

Из вышедоказанной леммы следует, что решение $\varphi$ линейной системы
\begin{equation*}
    \dot{r} = P(t)r + q(t)
\end{equation*}
устойчиво (асимптотически устойчиво), если и только если устойчиво (асимптотически устойчиво) решение $r = 0$ соответствующей линейной однородной системы
\begin{equation*}
    \dot{r} = P(t)r
\end{equation*}
Отсюда, в частности, вытекает, что все решения линейной системы имеют одинаковый характер устойчивости, поэтому можно говорить об устойчивости линейной системы.\\

\noindent \textbf{Теорема (Устойчивость ЛОС с постоянными коэффициентами).} Пусть $A \in M_n(\mathbb{R})$. Тогда система
\begin{equation}
    \dot{r} = Ar \label{lodnpost2}
\end{equation}
\begin{enumerate}
    \item асимптотически устойчива, если $Re \lambda < 0$ для всех $\lambda \in spec\, A$
    \item устойчива, если для каждого $\lambda \in spec\, A$ либо $Re\lambda < 0$, либо $Re \lambda = 0$ и алгебраическая кратность числа $\lambda$ совпадает с геометрической
    \item неустойчива, если найдется $\lambda \in spec\, A$, такое что либо $Re \lambda > 0$, либо $Re\lambda = 0$ и алгебраическая кратность числа $\lambda$ больше геометрической
\end{enumerate}
\noindent \textbf{Доказательство.} Через $T$ обозначим матрицу перехода к жорданову базису матрицы $A$, $J$ --- жорданова форма матрицы $A$.

Любое решение системы (\ref{lodnpost2}) имеет вид
\begin{equation*}
    r(t) = e^{At}r(0) = Te^{Jt}T^{-1}r(0)
\end{equation*}
Принимая во внимание лемму о норме матрицы, получаем
\begin{equation}
    |r(t)| \le n^3 |T||e^{Jt}||T^{-1}||r(0)| = K |e^{Jt}||r(0)| \label{norm}
\end{equation}
где $K > 0$ не зависит от $t$. Обозначим через $a_{ij}(t)$ элементы матрицы $e^{Jt}$.
\begin{enumerate}
    \item Каждая функция $a_{ij}(t)$ имеет вид $Ce^{\lambda t}t^k$, где $\lambda$ --- одно из собственных чисел. Поскольку $Re \lambda < 0$, то для всех $i, j$ будет $a_{ij}(t) \to 0$ при $t \to +\infty$, следовательно,
    \begin{equation*}
        |e^{Jt}| \to 0 \quad \text{при } t \to +\infty
    \end{equation*}
    Тогда из оценки (\ref{norm}) следует устойчивость и асимптотическая устойчивость нулевого решения.
    \item Каждая функция $a_{ij}(t)$ имеет вид $Ce^{\lambda t}t^k$, но теперь $k = 0$, если $Re \lambda = 0$. Следовательно, существует такое число $M$, что при всех $t \le 0$
    \begin{equation*}
        |e^{Jt}| < M
    \end{equation*}
    Поэтому из оценки (\ref{norm}) следует устойчивость нулевого решения.
    \item Пусть существует собственное число $\lambda = \alpha + i\beta$, где $\alpha > 0$. Обозначим через $h$ соответствующий собственный вектор. Тогда вектор-функция
    \begin{equation*}
        \varphi(t) = e^{\lambda t}h
    \end{equation*}
    является комплексным решением (\ref{lodnpost2}).
    
    Если же имеется собственное число $\lambda = i\beta$, $\beta \neq 0$, алгебраическая кратность которого больше геометрической, то система (\ref{lodnpost2}) имеет комплексное решение
    \begin{equation*}
        \varphi(t) = e^{i\beta t}(th_1 + h_2)
    \end{equation*}
    где $h_1, h_2$ --- собственный и присоединенный вектор, соответствующие число $\lambda$.
    
    Заметим, что в обоих случаях $|\varphi(t)| \to +\infty$ при $t \to +\infty$. Следовательно, хотя бы одна из вектор-функций, $Re\varphi(t)$ или $Im\varphi(t)$, является неограниченным вещественным решением (\ref{lodnpost2}). Пусть таковой является $Re\varphi(t)$. Тогда выбирая в качестве начального условия
    \begin{equation*}
        r(0) = \mu Re\varphi(0)
    \end{equation*}
    при достаточно малом $\mu$, получаем неограниченное решение с начальным значением в сколь угодно малой окрестности нуля. Отсюда следует, что нулевое решение неустойчиво.
\end{enumerate}
