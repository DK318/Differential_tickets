\textbf{Теорема (Общее решение ЛОС).} Пусть $P \in M_n(C(a,b))$. Тогда множество решений системы $\dot{r} = P(t)r$ образуют $n$-мерное линейное пространство.\\

\noindent \textbf{Доказательство.} Пусть $t_0 \in (a,b)$, $\{a_k\}_{k=1}^n$ --- базис в $\mathbb{R}^n$. Тогда для любого $k \in [1 : n]$ существует $r_k$ --- решение задачи Коши $\dot{r} = P(t)r,\, r(t_0) = a_k$. Вронскиан этих решений $W(t_0) = det(a_1, a_2, \ldots, a_n) \neq 0$. Тогда функции $\{r_k\}_{k=1}^n$ линейно независимы.

Рассмотрим произвольное решение $r$ системы $\dot{r} = P(t)r$. Пусть $\{c_k\}_{k=1}^n$ --- координаты вектора $r(t_0)$ в базисе $\{a_k\}_{k=1}^n$. Положим
\begin{equation*}
    \varphi = c_1r_1 + c_2r_2 + \ldots + c_nr_n
\end{equation*}
Ясно, что $\varphi$ --- решение системы $\dot{r} = P(t)r$, при этом $\varphi(t_0) = r_0$. Тогда $r \equiv \varphi$ в силу теоремы о единственности максимального решения ЛС.

Таким образом, функции $\{r_k\}_{k=1}^n$ линейно независимы, и любое решение есть их линейная комбинация. Значит, $\{r_k\}_{k=1}^n$ --- базис в пространстве решений.\\

\noindent \textbf{Определение.} \textbf{Фундаментальной системой решений} системы уравнений $\dot{r} = P(t)r$ называется совокупность ее $n$ линейно независимых решений.\\

\noindent \textbf{Определение.} \textbf{Фундаментальная матрица системы} $\dot{r} = P(t)r$ --- матрица, столбцы которой образуют фундаментальную систему решений.\\

\noindent \textbf{Лемма (о множестве фундаментальных матриц).} Пусть $\Phi$ --- фундаментальная матрица системы $\dot{r} = P(t)r$. Тогда $\left\{\Phi A \, | \, A \in M_n(\mathbb{R}), \, detA \neq 0 \right\}$ --- множество всех фундаментальных матриц этой системы.\\

\noindent \textbf{Доказательство.} Пусть $\Psi$ --- фундаментальная матрица системы $\dot{r} = P(t)r$. Тогда каждый ее столбец, будучи решением этой системы, является линейной комбинацией столбцов матрицы $\Phi$. Записывая коэффициенты разложения в столбцы матрицы $A$, имеем $\Psi = \Phi A$. А так как $det \Psi \neq 0$ и $\det \Phi \neq 0$, то и $det A \neq 0$.

Обратно, пусть $A \in M_n(\mathbb{R})$ --- произвольная невырожденная матрица. Тогда матрица $\Phi A$ состоит из решений, а ее определитель не обращается в ноль. Следовательно, эти решения линейно независимы, поэтому $\Phi A$ --- фундаментальная матрица.\\

\noindent \textbf{Лемма (Об овеществлении).} Пусть $n \in \mathbb{N}$, $\Phi = (r_1,r_2,r_3,\ldots,r_n)$ --- фундаментальная матрица системы $\dot{r} = P(t)r$, при этом $r_1 = \overline{r}_2$. Тогда
\begin{equation*}
    \Psi = (Re\,r_1, Im\,r_1, r_3, \ldots, r_n)
\end{equation*}
--- фундаментальная матрица той же системы.\\

\noindent \textbf{Доказательство.} Так как
\begin{equation*}
    \begin{aligned}
        &Re\, r_1 = \frac{1}{2}(r_1 + \overline{r}_1) = \frac{1}{2}r_1 + \frac{1}{2}r_2\\
        &Im\,r_1 = \frac{1}{2i}(r_1 - \overline{r}_1) = \frac{1}{2i}r_1 - \frac{1}{2i}r_2
    \end{aligned}
\end{equation*}
то
\begin{equation*}
    \Psi = \Phi \begin{pmatrix}
    \begin{matrix}
    \frac{1}{2} & \frac{1}{2i}\\
    \frac{1}{2} & -\frac{1}{2i}
    \end{matrix} & 0\\
    0 & E_{n-2}
    \end{pmatrix}
\end{equation*}
где $E_{n-2}$ --- единичная матрица порядка $n - 2$. По лемме о множестве фундаментальных матриц матрица $\Psi$ является фундаментальной.
