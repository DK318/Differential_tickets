Рассмотрим нелинейную автономную систему $\dot{r} = f(r)$. Допустим, вектор-функция $f$ дифференцируема. Тогда по формуле Тейлора
\begin{equation*}
    f(r) = f(0) + f'(0)r + o(r)
\end{equation*}

\noindent \textbf{Определение.} Пусть $f(0) = 0$. Тогда система
\begin{equation*}
    \dot{r} = f'(0)r
\end{equation*}
называется \textbf{системой первого приближения} или \textbf{линеаризацией} системы $\dot{r} = f(r)$\\

\noindent \textbf{Теорема (Ляпунов, устойчивость по первому приближению).} Пусть $f \in C^2(\Omega)$, где $\Omega \subset \mathbb{R}^n$ --- окрестность нуля, $f(0) = 0$. Тогда нулевое решение системы $\dot{r} = f(r)$
\begin{enumerate}
    \item асимптотически устойчиво, если $Re\, \lambda < 0$ для любого $\lambda \in spec\, f'(0)$
    \item неустойчиво, если найдется $\lambda \in spec\, f'(0)$, для которого $Re\, \lambda > 0$
\end{enumerate}

\noindent \textbf{Доказательство.} См. конспект, на консультации Бабушкин сказал, что на лекции теорема была без док-ва ибо он сам не помнит, как доказывать.\\

\noindent \textbf{Определение.} Пусть $\Omega \subset \mathbb{R}^n$ --- окрестность нуля. Функция $V \in C^1(\Omega)$ называется \textbf{функцией Ляпунова} системы $\dot{r} = f(r)$, если
\begin{itemize}
    \item $V(r) > 0$ при всех $r \in \Omega \setminus \{0\}$, $V(0) = 0$
    \item $V' \cdot f \le 0$ при всех $r \in \Omega$
\end{itemize}

\noindent \textbf{Теорема (Ляпунов, об устойчивости).} Пусть $f \in Lip_{loc}(\Omega)$, $\Omega \subset \mathbb{R}^n$ --- окрестность нуля, $f(0) = 0$. Если в области $\Omega$ существует функция Ляпунова системы $\dot{r} = f(r)$, то $r = 0$ --- устойчивое решение.\\

\noindent \textbf{Доказательство.} Будем доказывать от противного. Пусть нулевое положение равновесия неустойчиво. Тогда найдется такое $\varepsilon > 0$, что при любом сколь угодно малом $\delta > 0$ можно выбрать такое начальное положение $r_0$ из $\delta$-окрестности нуля, что будет ложным утверждение
\begin{equation}
    \forall t \ge 0 \quad |r(t, 0, r_0)| < \varepsilon \label{lyap}
\end{equation}
Все числа, меньшие $\varepsilon$, обладают тем же свойством, что и $\varepsilon$. Поэтому можно считать, что $\varepsilon$-окрестность содержится вместе с границей в области $\Omega$.

Ложность утверждения (\ref{lyap}) означает одно из двух:
\begin{enumerate}
    \item решение $r(t, 0, r_0)$ определено не при всех $t \ge 0$
    \item найдется $t_{\varepsilon} > 0$, такое что $|r(t_{\varepsilon}, 0, r_0)| \ge \varepsilon$
\end{enumerate}

Допустим, выполнено (1), то есть максимальное решение $r(t, 0, r_0)$ (которое существует и единственно) определено на интервале $(a,b) \ni 0$, где $b < +\infty$. Построим параллелепипед $[0,b] \times \overline{B}_{\varepsilon}(0)$, где $\overline{B}_{\varepsilon}(0)$ --- замыкание $\varepsilon$-окрестности нуля. Интегральная кривая решения $r(t, 0, r_0)$ выйдет на его границу при некотором $t_{\varepsilon} < b$. Значит, верно утверждение (2).

Итак, достаточно получить противоречие, если верно (2). Будем считать, что $t_{\varepsilon}$ --- это точка, в которой $r(t_{\varepsilon}, 0, r_0) \in \partial B_{\varepsilon}(0)$, где $\partial B_{\varepsilon}(0)$ --- граница $\varepsilon$-окрестности нуля.

Пусть $V_M = \displaystyle\min_{r \in \partial B_{\varepsilon}(0)}V(r)$. Поскольку $V(r) \to 0$ при $r \to 0$, то можно выбрать $\delta$ так, чтобы $V(r) < \frac{V_M}{2}$ при $|r| < \delta$. Положим
\begin{equation*}
    v(t) = V(r(t, 0, r_0))
\end{equation*}
где $r_0$ выбрано из $\delta$-окрестности нуля так, чтобы выполнялось (2).

Функция $v$ дифференцируема
\begin{equation*}
    \begin{aligned}
        &v(0) = V(r_0) < \frac{V_M}{2}\\
        &v(t_{\varepsilon}) = V(r(t_{\varepsilon}, 0, r_0)) \ge V_M
    \end{aligned}
\end{equation*}
Поэтому найдется точка $t_1 \in (0, t_{\varepsilon})$, в которой $\dot{v}(t_1) > 0$.

Однако, исходя из второго свойства функции Ляпунова, имеем
\begin{equation*}
    \dot{v}(t_1) = V'(r(t_1, 0, r_0))\cdot r'_t(t_1, 0, r_0) = V'(r(t_1, 0, r_0)) \cdot f(r(t_1, 0, r_0)) \le 0
\end{equation*}
Полученное противоречие завершает доказательство теоремы.\\

\noindent \textbf{Теорема (Ляпунов, об асимптотической устойчивости).} Если в условиях предыдущей теоремы выполнено $V' \cdot f < 0$ в области $\Omega \setminus \{0\}$, то нулевое решение системы $\dot{r} = f(r)$ асимптотически устойчиво.
