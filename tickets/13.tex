\textbf{Теорема.} Пусть $n \in \mathbb{N}$, $G \subset \mathbb{R}_{t,r}^{n+1}$ --- область, $f \in C(G \to \mathbb{R}^n)\cap Lip_{r,loc}(G)$, $\varphi$ --- максимальное решение на $(a,b)$ уравнения $\dot{r} = f(t,r)$, $K \subset G$ --- компакт. Тогда найдется $\Delta > 0$, такое что $(t, \varphi(t)) \notin K$ при всех $t \in (a, a + \Delta) \cup (b - \Delta, b)$.\\

\noindent \textbf{Доказательство.} Заметим, что расстояние $\rho = \rho(K, \partial G)$ от компакта $K$ до границы $\partial G$ области $G$ положительно (иначе можно было бы построить последовательность точек из $K$, сходящейся к точке на границе, но $\partial G \cap K = \varnothing$). Если $\rho < +\infty$, положим $c = \frac{\rho}{2}$, иначе пусть $c = 1$.

Вокруг каждой точки $(t', r') \in K$ построим содержащийся внутри $G$ параллелепипед
\begin{equation*}
    \Pi(t',r') = \left\{(t,r) \in \mathbb{R}^{n+1}\, | \, |t-t'| \le c,\, |r - r'| \le c \right\}
\end{equation*}
и рассмотрим множество
\begin{equation*}
    K_c = \bigcup_{(t',r') \in K} \Pi(t',r')
\end{equation*}

Поскольку $K$ --- компакт, то норма каждого элемента из $K$ ограничена некоторым числом $d$. Если $(t,r)$ --- произвольная точка из $K_c$, то для некоторой точки $(t',r') \in K$ будет $(t,r) \in \Pi(t',r')$, поэтому
\begin{equation*}
    |(t,r)| \le |(t,r) - (t',r')| + |(t',r')| \le c + d
\end{equation*}
Значит, множество $K_c$ ограничено.

Докажем его замкнутость. Рассмотрим последовательность $\{(t_{m_k}, r_{m_k})\}$ точек из $K_c$, сходящуюся к $(t,r) \in \mathbb{R}^{n+1}$. Для каждой такой точки найдется параллелепипед $\Pi(t'_{m_k},r'_{m_k})$, которому она принадлежит. Раз $K$ --- компакт, то существует подпоследовательность $\{(t'_{m_k}, r'_{m_k})\}$, сходящаяся к некоторой точке $(t',r') \in K$. Переходя к пределу в неравенствах
\begin{equation*}
    |t_{m_k} - t'_{m_k}| \le c, \quad |r_{m_k} - r'_{m_k}| \le c
\end{equation*}
находим $|t - t'| \le c$ и $|r - r'| \le c$. Следовательно $(t,r) \in K_c$.

Таким образом, $K_c$ --- компакт, и функция $f$ достигает на нем максимального значения
\begin{equation*}
    M = \max_{(t,r) \in K_c} |f(t,r)|
\end{equation*}

Теперь предположим, что утверждение теоремы неверно. Пусть $\Delta = \frac{h}{2}$, где $h = \min \{c, \frac{c}{M}\}$. Тогда при некотором $t_0 \in (b - \frac{h}{2}, b)$ будет $(t_0, \varphi(t_0)) \in K$.

Рассмотрим ЗК $\dot{r} = f(t,r),\, r(t_0) = \varphi(t_0)$. По теореме Пеано она имеет решение $\psi$ на отрезке $[t_0 - h, t_0 + h]$. Пусть
\begin{equation*}
    \widetilde{\varphi}(t) = \begin{cases}
    \varphi(t), \quad \text{если } t \in (a, t_0)\\
    \psi(t), \quad \text{если } t \in [t_0, t_0 + h]
    \end{cases}
\end{equation*}

По лемме о гладкой стыковке решений $\widetilde{\varphi}$ --- решение уравнения $\dot{r} = f(t,r)$ на $(a, t_0 + h)$. Функция $\widetilde{\varphi} \equiv \varphi$ на $(a,b) \cap (a, t_0 + h)$ по теореме Пикара. Но
\begin{equation*}
    t_0 + h > b - \frac{h}{2} + h = b + \frac{h}{2} > b
\end{equation*}
то есть $\widetilde{\varphi}$ --- продолжение $\varphi$ вправо за точку $b$. Так как $\varphi$ по условию является максимальным решением, приходим к противоречию.
