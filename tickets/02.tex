\label{bil2}
\textbf{Определение.} Дифференциальное уравнение вида
\begin{equation}
    y' = p(x)y + q(x) \label{lnu}
\end{equation}
называется \textbf{линейным уравнением первого порядка.}\\

\noindent \textbf{Определение.} Уравнение (\ref{lnu}) называется \textbf{однородным}, если $q \equiv 0$ на $(a,b)$, иначе \textbf{неоднородным}.\\

\hypertarget{loulemma}
\noindent \textbf{Лемма (общее решение ЛОУ 1-го порядка).} Пусть $p \in C(a,b)$. Тогда общее решение уравнения
\begin{equation}
    y' = p(x)y \label{lou}
\end{equation}
имеет вид
\begin{equation*}
    y = Ce^{\int p},\quad dom\,y = (a,b), \quad C \in \mathbb{R}
\end{equation*}
\noindent \textbf{Доказательство.} Заметим, что функция, тождественно равная нулю на $(a,b)$, является решением. По теореме о единственности решения ЗК оно не является особым. В области, где $y > 0$, исходное уравнение равносильно
\begin{equation*}
    \frac{dy}{y} = p(x)dx
\end{equation*}
Интегрируя, получаем
\begin{equation*}
    \ln y = \int p(x)dx + C
\end{equation*}
Отсюда
\begin{equation*}
    y = Ae^{\int p(x)dx},\, A > 0
\end{equation*}
По теореме об общем решении УРП полученное соотношение описывает все интегральные кривые в области, где $y > 0$.

Аналогично с $y < 0$. Таким образом, все решения имеют требуемый вид.\\

\noindent \textbf{Теорема (Общее решение ЛУ 1-го порядка).} Пусть $p,q \in C(a,b)$. Тогда общее решение уравнения (\ref{lnu}) имеет вид
\begin{equation}
    y = (C + \int qe^{-\int p})e^{\int p} \label{lnusol}
\end{equation}
\textbf{Доказательство.} Подстановкой легко убеждаемся, что указанная функция при любом $C$ является решением на $(a,b)$.

Проверим, что нет других решений. Докажем от противного. Пусть имеется решение $\varphi$ на $(\alpha, \beta) \subset (a,b)$, не определяемое формулой (\ref{lnusol}) ни при каком $C$. Пусть еще $x_0 \in (\alpha, \beta)$, $\varphi(x_0) = y_0$. При
\begin{equation*}
    C = \left[y_0e^{-\int p(x)dx} - \int q(x)e^{-\int p(x)dx}dx\right]_{x=x_0}
\end{equation*}
функция определена на $(\alpha, \beta)$ и является решением ЗК для уравнения (\ref{lnu}) с начальным условием $y(x_0) = y_0$. Тогда $y \equiv \varphi$ на $(\alpha, \beta)$ по теореме о единственности решения ЗК, что противоречит предположению о функции $\varphi$.

Существует также \textbf{метод Лагранжа}, который позволяет решить уравнение (\ref{lnu}). Он состоит в следующем. По \hyperlink{loulemma}{лемме} решение соответсвующего однородного уравнения имеет вид
\begin{equation}
    y = Ce^{\int p} \label{lousol2}
\end{equation}
где $C$ --- произвольная постоянная. Это выражение подставляют в исходное уравнение (\ref{lnu}), при этом считая $C$ функцией от $x$. Получается уравнение относительно функции $C$:
\begin{equation*}
    C' = qe^{-\int p}
\end{equation*}
Его решение совпадает с выражением в скобках в (\ref{lnusol}). Поэтому общее решение исходного уравнения (\ref{lnu}) получится, если найденную функцию $C$ подставить вместо постоянной $C$ в общем решении (\ref{lousol2}) однородного уравнения.
