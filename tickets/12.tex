\textbf{Теорема.} Пусть $G \subset \mathbb{R}_{t,r}^{n+1}$ --- область, $f \in C(G \to \mathbb{R}^{n}) \cap Lip_{r,loc}(G)$, $(t_0, r_0) \in G$. Тогда максимальное решение ЗК $\dot{r} = f(t,r),\, r(t_0) = r_0$ существует и единственно.\\

\noindent \textbf{Доказательство.} Рассмотрим множество $S$ всевозможных решений ЗК, определенных на интервалах. По теореме Пеано это множество не пусто. Обозначим через $(a_{\varphi}, b_{\varphi})$ область определения решения $\varphi \in S$. Положим
\begin{equation*}
    (A,B) = \bigcup_{\varphi \in S}(a_{\varphi}, b_{\varphi})
\end{equation*}

Определим на $(A,B)$ функцию $\psi$ следующим образом. Если $t \in (A,B)$, то найдется функция $\varphi \in S$, такая что $t \in (a_{\varphi}, b_{\varphi})$. Тогда положим $\psi(t) = \varphi(t)$. Определение корректно, поскольку значение в точке $t$ любой другой функции из $S$, определенной в $t$, совпадает с $\varphi(t)$ по теореме Пикара.

Отсюда следует, что $\psi \equiv \varphi$ на $(a_{\varphi}, b_{\varphi})$. А раз $\varphi$ --- решение, то $\psi$ непрерывно дифференцируема в $t$ и $\dot{\psi}(t) = \dot{\varphi}(t) = f(t,\varphi(t)) = f(t,\psi(t))$. В силу произвольности выбора точки $t$ получаем
\begin{itemize}
    \item $\psi \in C^1(A,B)$
    \item $\dot{\psi} = f(t,\psi(t))$ при всех $t \in (A,B)$
\end{itemize}
Кроме того, $\psi(t_0) = \varphi(t_0) = r_0$. Тогда $\psi$ --- решение исходной ЗК по определению.

Поскольку интервал $(A,B)$ включает в себя все возможные интервалы, на которых могут быть заданы решения, то $\psi$ является максимальным решением.

Другого максимального решения быть не может. Докажем от противного. Пусть имеется еще одно максимальное решение $\widetilde{\psi}:\, (\widetilde{A}, \widetilde{B}) \to \mathbb{R}^n$. Тогда $(\widetilde{A}, \widetilde{B}) \subset (A,B)$ и $\widetilde{\psi} \equiv \psi$ на $(\widetilde{A}, \widetilde{B})$. Если, например, $\widetilde{B} < B$, то $\psi$ --- продолжение $\widetilde{\psi}$ вправо, что противоречит непродолжимости решения $\widetilde{\psi}$.