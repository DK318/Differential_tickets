\textbf{Определение.} Определим отображение $\Lambda_n y$ формулой
\begin{equation*}
    \Lambda_n y = (y, \dot{y},\ldots, y^{(n-1)})^T
\end{equation*}

\noindent \textbf{Теорема (Существование и единственность решения ЗК для уравнения высшего порядка)}. Пусть $G \subset \mathbb{R}_{t, y, y', \ldots, y^{(n-1)}}^{n + 1}$ --- область, $f \in C(G) \cap Lip_{(y,y',\ldots,y^{(n-1)}),loc}(G)$, $(t_0, y_0, y'_0, \ldots, y^{(n-1)}_0) \in G$. Тогда
\begin{enumerate}
    \item в некоторой окрестности точки $t_0$ существует решение задачи
    \begin{equation}
        \begin{cases}
            y^{(n)} = f(t, y, \dot{y}, \ldots, y^{(n-1)})\\
            y(t_0) = y_0,\,\dot{y}(t_0) = y'_0, \ldots, y^{(n-1)}(t_0) = y_0^{(n-1)}
        \end{cases} \label{vish}
    \end{equation}
    \item если $\psi_1$ и $\psi_2$ --- решения (\ref{vish}) на $(a,b)$, то $\psi_1 \equiv \psi_2$ на $(a,b)$
\end{enumerate}
\noindent \textbf{Доказательство.} Пусть $y$ --- решение (\ref{vish}) на $(a,b)$. По лемме о равносильной системе вектор-функция $\Lambda_n y$ --- решение ЗК для равносильной системы
\begin{equation}
    \begin{pmatrix}
    \dot{y}_1\\
    \ldots\\
    \dot{y}_{n-1}\\
    \dot{y}_n
    \end{pmatrix}
    =
    \begin{pmatrix}
    y_2\\
    \ldots\\
    y_n\\
    f(t, y_1, y_2, \ldots, y_n)
    \end{pmatrix}
    \label{ravnsystcyka}
\end{equation}
с начальными условиями
\begin{equation}
    y_1(t_0) = y_0,\, y_2(t_0) = y'_0,\ldots, y_n(t_0) = y_0^{(n-1)} \label{nach}
\end{equation}

Правая часть каждого уравнения системы (\ref{ravnsystcyka}) непрерывна и удовлетворяет условию Липшица локально по переменным $y, \dot{y},\ldots, y^{(n-1)}$ в области $G$. Тогда по теореме Пикара соответствующая задача Коши не может иметь более одного решения на $(a,b)$, а значит, не может быть более одного решения и у задачи (\ref{vish}). Тем самым доказан пункт (2).

Установим справедливость пункта (1). Из теоремы Пикара следует существование в некоторой окрестности точки $t_0$ решения $r = (y_1, \ldots, y_n)$ системы (\ref{ravnsystcyka}) с начальными условиями (\ref{nach}). По лемме о равносильной системе функция $y_1$ --- решение задачи (\ref{vish}) в той же окрестности.\\

\noindent \textbf{Следствие (Теорема Пикара с простыми условиями).} Пусть $G \subset \mathbb{R}_{t,r}^{n+1}$ --- область, $f \in C(G \to \mathbb{R}^n)$, $f'_r \in M_n(C(G))$, $(t_0, r_0) \in G$. Тогда
\begin{enumerate}
    \item на отрезке Пеано $[t_0 - h, t_0 + h]$ существует решение $\varphi$ ЗК
    \item если $\psi_1$ и $\psi_2$ --- решения этой задачи на $(a,b)$, то $\psi_1 \equiv \psi_2$ на $(a,b)$
    \item пусть $\Pi$ --- параллелепипед, по которому строится отрезок Пеано,
    \begin{equation*}
        M = \max_{(t,r) \in \Pi} |f(t,r)|, \quad M_1 = \max_{(t,r) \in \Pi} |f'_r(t,r)|
    \end{equation*}
    $\varphi_m$ --- $m$-е приближение Пикара, тогда для любого $t \in [t_0 - h, t_0 + h]$
    \begin{equation*}
        |\varphi(t) - \varphi_m(t)| \le \frac{M(nM_1)^mh^{m+1}}{(m+1)!}
    \end{equation*}
\end{enumerate}
\textbf{Доказательство.} Пункты (1) и (2) следуют из леммы о локальном условии Липшица и теоремы Пикара. Для доказательства пункта (3) сделаем предельный переход в неравенстве (\ref{indh}) при $k \to \infty$. Тогда
\begin{equation*}
    |\varphi(t) - \varphi_m(t)| \le \frac{ML^mh^{m+1}}{(m+1)!}
\end{equation*}
По лемме о локальном условии Липшица в качестве $L$ можно взять $nM_1$.

