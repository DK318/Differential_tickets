\textbf{Теорема (Пикар).} Пусть $G \subset \mathbb{R}_{t,r}^{n+1}$ --- область, $f \in C(G \to \mathbb{R}^n) \cap Lip_{r,loc}(G)$, $(t_0, r_0) \in G$. Тогда
\begin{enumerate}
    \item на отрезке Пеано существует решение задачи
    \begin{equation}
        \dot{r} = f(t,r), \quad r(t_0) = r_0 \label{zkpeano3}
    \end{equation}
    \item если $\psi_1$ и $\psi_2$ --- решения (\ref{zkpeano3}) на $(a,b)$, то $\psi_1 \equiv \psi_2$ на $(a,b)$
\end{enumerate}

\noindent \textbf{Следствие (Теорема Пикара с простыми условиями).} Пусть $G \subset \mathbb{R}_{t,r}^{n+1}$ --- область, $f \in C(G \to \mathbb{R}^n)$, $f'_r \in M_n(C(G))$, $(t_0, r_0) \in G$. Тогда
\begin{enumerate}
    \item на отрезке Пеано $[t_0 - h, t_0 + h]$ существует решение $\varphi$ ЗК
    \item если $\psi_1$ и $\psi_2$ --- решения этой задачи на $(a,b)$, то $\psi_1 \equiv \psi_2$ на $(a,b)$
    \item пусть $\Pi$ --- параллелепипед, по которому строится отрезок Пеано,
    \begin{equation*}
        M = \max_{(t,r) \in \Pi} |f(t,r)|, \quad M_1 = \max_{(t,r) \in \Pi} |f'_r(t,r)|
    \end{equation*}
    $\varphi_m$ --- $m$-е приближение Пикара, тогда для любого $t \in [t_0 - h, t_0 + h]$
    \begin{equation*}
        |\varphi(t) - \varphi_m(t)| \le \frac{M(nM_1)^mh^{m+1}}{(m+1)!}
    \end{equation*}
\end{enumerate}
\textbf{Доказательство.} Пункты (1) и (2) следуют из леммы о локальном условии Липшица и теоремы Пикара. Для доказательства пункта (3) сделаем предельный переход в неравенстве (\ref{indh}) при $k \to \infty$. Тогда
\begin{equation*}
    |\varphi(t) - \varphi_m(t)| \le \frac{ML^mh^{m+1}}{(m+1)!}
\end{equation*}
По лемме о локальном условии Липшица в качестве $L$ можно взять $nM_1$.

