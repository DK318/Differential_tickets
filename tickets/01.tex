\textbf{Определение.} Уравнение в дифференциалах вида.
\begin{equation}
  X(x)dx + Y(y)dy = 0 \label{formula1}
\end{equation}
называют \textbf{уравнением с разделенными переменными.}\\
\\
\textbf{Теорема (общее решение УРП).} Пусть $X \in C(a,b)$, $Y \in C(c,d)$, $(\alpha, \beta) \subset (a,b)$. Тогда
\begin{enumerate}
    \item если $y$ --- решение на $(\alpha, \beta)$ уравнения (\ref{formula1}), то при некотором значении $C \in \mathbb{R}$ функция $y$ неявно задается уравнением
    \begin{equation}
        \int X(x)dx + \int Y(y)dy = C \label{formula2}
    \end{equation}
    \item если при некотором значении $C \in \mathbb{R}$ уравнение (\ref{formula2}) неявно задает функцию $y \in C^1(\alpha, \beta)$, то $y$ --- решение уравнения (\ref{formula1}) на $(\alpha, \beta)$. 
\end{enumerate}
\textbf{Доказательство}
\begin{enumerate}
    \item Покажем, что функция $y(x)$ удовлетворяет уравнению (\ref{formula2}) при любом $x \in (\alpha, \beta)$ и некотором $C \in \mathbb{R}$. Имеем
    \begin{equation*}
        \int X(x)dx + \int Y(y)dy = \int X(x)dx + \int Y(y)y'dx = \int (X(x) + Y(y)y')dx
    \end{equation*}
    Так как $y$ --- решение уравнения (\ref{formula1}) на $(\alpha, \beta)$, то $\forall x \in (\alpha, \beta)$ подынтегральное выражение равно нулю $\implies$ интеграл равен некоторой постоянной.
    \item Покажем, что функция $y(x)$ удовлетворяет уравнению (\ref{formula1}) $\forall x \in (\alpha, \beta)$. Дифференцируя обе части (\ref{formula2}), получим
    \begin{equation*}
        X(x) + Y(y)y' = 0
    \end{equation*}
    Исходя из условия, это равенство выполнено тождественно на $(\alpha, \beta)$ $\implies$ $y$ --- решение уравнения (\ref{formula1}) по определению.
\end{enumerate}
\\
\textbf{Определение.} Уравнение вида
\begin{equation}
    p_1(x)q_1(y)dx + p_2(x)q_2(y)dy = 0 \label{formula3}
\end{equation}
называют \textbf{уравнением с разделяющимися переменными}.\\

При делении на $q_1(y)p_2(x)$ уравнение приводится к уравнению с разделенными переменными. Необходимо убедиться, что не происходит деления на ноль.

Пусть $p_1, p_2 \in C(a,b)$ и $q_1, q_2 \in C(c,d)$. Если $q_1(y_0) = 0$, то $y \equiv y_0$, $x \in (a,b)$ --- решение исходного уравнения. Для поиска других интегральных кривых требуется разбить область задания уравнения на две подобласти, общей границей которых является прямая $y = y_0$.

Аналогично следует поступить для $p_2(x_0) = 0$. В таком случае $x \equiv x_0$, $y \in (c,d)$ --- решение исходного уравнения.

Разбив всю область на необходимое количество частей, нужно рассмотреть исходное уравнение на каждой части отдельно. На каждой такой подобласти можно разделить исходное уравнение на $q_1(y)p_2(x)$, не опасаясь получить ноль в знаменателе.

Изучив поведение интегральных кривых вблизи границ, делается вывод о наличии особых и составных решений уравнения на исходной области его задания.\\

\textbf{Теорема (существование и единственность решения задачи Коши для УРП)}. Пусть $X \in C(a,b)$, $Y \in C(c,d)$, $(x_0, y_0)$ --- не особая точка уравнения (\ref{formula1}). Тогда в некоторой окрестности точки $(x_0, y_0)$ уравнение
\begin{equation*}
    \int_{x_0}^{x} X(s)ds + \int_{y_0}^{y} Y(s)ds = 0
\end{equation*}
определяет единственную интегральную кривую уравнения (\ref{formula1}), проходящую через точку $(x_0, y_0)$.
