\textbf{Теорема (Общее решение ЛНС).} Пусть $P \in M_n(C(a,b))$, $q \in C((a,b) \to \mathbb{R}^n)$, $\varphi$ --- решение системы
\begin{equation}
    \dot{r} = P(t)r + q(t) \label{lnusyst2}
\end{equation}
$\Phi$ --- фундаментальная матрица системы
\begin{equation*}
    \dot{r} = P(t)r
\end{equation*}
Тогда общее решение неоднородной системы (\ref{lnusyst2}) имеет вид
\begin{equation*}
    r = \Phi C + \varphi, \quad C \in \mathbb{R}^n
\end{equation*}
\textbf{Доказательство.} Пусть $r$ --- произвольное решение (\ref{lnusyst2}). Тогда
\begin{equation*}
    \dot{r} = Pr + q
\end{equation*}
Функция $\varphi$ удовлетворяет такому же соотношению:
\begin{equation*}
    \dot{\varphi} = P\varphi + q
\end{equation*}
Вычитая эти равенства, находим
\begin{equation*}
    (r - \varphi)' = P(r - \varphi)
\end{equation*}
Значит, найдется вектор-столбец $C \in \mathbb{R}^n$, такой что
\begin{equation*}
    r - \varphi = \Phi C
\end{equation*}

Верно и обратное: любая функция вида $\Phi C + \varphi$ являются решением (\ref{lnusyst2}), что проверяется непосредственной подстановкой.\\

\noindent \textbf{Теорема (метод вариации постоянных для ЛНС).} Пусть $\Phi$ --- фундаментальная матрица системы $\dot{r} = P(t)r$, $P \in M_n(C(a,b))$, $q \in C((a,b) \to \mathbb{R}^n)$. Тогда если вектор-функция $C$ пробегает все решения системы
\begin{equation*}
    \Phi\dot{C} = q
\end{equation*}
то $r = \Phi C$ пробегает все решения системы (\ref{lnusyst2})\\

\noindent \textbf{Доказательство.} Опираясь на формулу для обратной матрицы, использующей алгебраические дополнения, заключаем, что $\Phi^{-1} \in M_n(C(a,b))$. Поэтому
\begin{equation*}
    C(t) = \int \Phi^{-1}q + A
\end{equation*}
где $A$ --- вектор произвольных постоянных. Тогда требуется доказать, что общее решение системы (\ref{lnusyst2}) имеет вид
\begin{equation*}
    r = \Phi A + \Phi \int \Phi^{-1}q
\end{equation*}
По теореме об общем решении ЛНС достаточно показать, что второе слагаемое в правой части --- частное решение системы (\ref{lnusyst2}). Убедимся в этом подстановкой:
\begin{equation*}
    \dot{\Phi}\int \Phi^{-1}q + \Phi\Phi^{-1}q = P(t)\Phi \int \Phi^{-1}q + q
\end{equation*}
Это верное тождество, поскольку $P(t)\Phi = \dot{\Phi}$.